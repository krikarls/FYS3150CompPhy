\documentclass[11pt,a4paper,twoside,reqno]{amsart}
\usepackage[utf8]{inputenc}
\usepackage[T1]{fontenc}
%\usepackage{babel}
%\usepackage{fancyhdr}
\usepackage{amsfonts,amsmath,amssymb,pifont,enumerate,graphicx,textcomp,varioref,algpseudocode,listings}
\usepackage[margin=1in,footskip=0.25in]{geometry}
%\usepackage[procnames]{listings}
\usepackage{color}
\usepackage{tcolorbox}
\title{FYS3150/4150 - Project 1}
\author{Krister Stræte Karlsen\\
\\ \texttt{}}
\tolerance = 5000 
\hbadness = \tolerance 
\pretolerance = 2000 

\newcommand{\R}{\mathbb{R}}
\newcommand{\Q}{\mathbb{Q}}
\renewcommand{\P}{\mathbb{P}}
\renewcommand{\leq}{\leqslant}
\renewcommand{\geq}{\geqslant}
\newcommand{\abs}[1]{\left\lvert #1 \right\rvert}
\newcommand{\intlim}[3]{\left. #1 \right\rvert_{ #2 }^{ #3 }}
\newcommand{\sequence}[3]{\left\{ #1 \right\}_{ #2 }^{ #3 }}
\newcommand{\set}[1]{\left\{ #1 \right\}}

\definecolor{keywords}{RGB}{0,0,160}
\definecolor{comments}{RGB}{100,100,100}
\definecolor{red}{RGB}{160,0,0}
\definecolor{green}{RGB}{50,102,0}
\definecolor{mygray}{RGB}{30,30,30}

\lstset{keywordstyle=\color{kewwords},backgroundcolor=\color{mygray},basicstyle=\footnotesize\ttfamily,breaklines=true,commentstyle=\color{mygray},language=Python,firstline=1,lastline=53,frame=single,numbers=left,numberstyle=\tiny\color{mygray}}


\begin{document}

 
\maketitle
\noindent
a)\\
\noindent
Given the following ODE with boundary conditions, 

\[
-u''(x) = f(x), \hspace{0.5cm} x\in(0,1), \hspace{0.5cm} u(0) = u(1) = 0.
\]

and the dicretized form

\[
   -\frac{v_{i+1}+v_{i-1}-2v_i}{h^2} = f_i  \hspace{0.5cm} \mathrm{for} \hspace{0.5cm} i=1,\dots, n, \hspace{0.1cm} v_0 = v_{n+1} = 0
\]

we are going to show it can be written as system of linear equations of the form: 

\[
   {\bf A}{\bf v} = \tilde{{\bf b}},
\]

\noindent
\textbf{Solution:}\\
Multipling the discretized equation by $h^2$ we get:

\[
   -v_{i-1}+2v_{i}-v_{i+1}=h^2 f_i \hspace{0.5cm} \mathrm{for} \hspace{0.5cm} i=1,\dots, n 
\]
\vspace{0.5cm}
\noindent
Filling in for $i$ and chosing $\tilde{b_i} = h^2 f_i$ we obtain the following set of equations: 

\begin{align*}
	2v_{1}-v_{2}=\tilde{b_1} \\
	-v_{1}+2v_{2}-v_{3}=\tilde{b_2} \\
	\vdots \hspace{0.5cm} \\
	-v_{i-1}+2v_{i}-v_{i+1}=\tilde{b_i} \\
	\vdots \hspace{0.5cm}  \\ 
	-v_{n-1}+2v_{n}=\tilde{b_n} \\
\end{align*}
\noindent
Now one can easily see that this system of linear equations can written as 

\[
   {\bf A}{\bf v} = \tilde{{\bf b}},
\]

where 
\begin{equation*}
    {\bf A} = \left(\begin{array}{cccccc}
                           2& -1& 0 &\dots   & \dots &0 \\
                           -1 & 2 & -1 &0 &\dots &\dots \\
                           0&-1 &2 & -1 & 0 & \dots \\
                           & \dots   & \dots &\dots   &\dots & \dots \\
                           0&\dots   &  &-1 &2& -1 \\
                           0&\dots    &  & 0  &-1 & 2 \\
                      \end{array} \right),\hspace{0.3cm} {\bf v}= \left(\begin{array}{c}
                           v_1\\
                           v_2\\
                           \dots \\
                          \dots  \\
                          \dots \\
                           v_n\\
                      \end{array} \right)
  ,\hspace{0.3cm} \tilde{{\bf b}}, = \left(\begin{array}{c}
                           \tilde{b}_1\\
                           \tilde{b}_2\\
                           \dots \\
                           \dots \\
                          \dots \\
                           \tilde{b}_n\\
                      \end{array} \right).
\end{equation*}


\end{document}



